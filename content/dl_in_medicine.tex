\chapter{Deep Learning in Digital Histopathology}
Manual analysis of histopathology slides is expensive, takes long time to complete and requires highly trained professionals and quality assurance by performing peer reviews \cite{Wemmert2021}. With the invention of virtual microscopy, which enables glass slides to be converted into digital slides, and the introduction of Whole-Slide Images (WSIs), the field is entering a new era. The term Digital Pathology is often being used. In Digital Pathology, much effort is put into developing tools that would help medical experts to semi- or fully automate the visual analysis of the digital slides. Entities such as different tissue types and cells can be identified and classified.

Deep learning has shown extreme potential in many areas, including medicine and processing of medical image data \cite{LeCun2015}.

\section{Architectures}
Neural network architectures like Convolutional Neural Networks \cite{LeCun2015-2} and U-Net \cite{Ronneberger2015} have proven to be effective in medical image analysis \cite{Santosh2022-2}. In recent years, also a concept of Vision Transformers \cite{Dosovitskiy2020, Hu2023} used in medical imaging shows promising results \cite{Shamshad2023, Hu2023, He2023}.

\subsection{Convolutional Neural Networks}

\subsection{U-Net}

\subsection{Vision Transformer}

\section{Challenges, Strategies and Future Directions}

\subsection{Data Annotation}

\subsection{Weakly Supervised Learning}

\subsection{Active Learning}

\subsection{Future Trends}

