
\thispagestyle{empty}

\section*{Annotation}

\begin{minipage}[t]{1\columnwidth}%
Slovak University of Technology, Bratislava 

Faculty of Informatics and Information Technologies

Degree Course: \myStudyProgram\\

Author: \myName

Bachelor's Thesis: \myTitle

Supervisor: \mySupervisor

\myDate%
\end{minipage}

\bigskip{}

%Breast cancer is one of the leading causes of death among women, and accurate histopathological analysis plays a significant role in diagnosis and treatment planning. Tumor-infiltrating lymphocytes (TILs) have emerged as a promising biomarker. However, manual analysis and counting of TILs in histopathological slides remain time-consuming, error-prone, and require skilled professionals. Deep learning models have shown promise in automating this process, but they depend on large amounts of high-quality annotated data. 

In this thesis, we look at state-of-the-art weak segmentation techniques in digital pathology, focusing on segmenting lymphocyte cell nuclei in breast cancer patients. The main challenge stems from the weak annotations of nuclei in the form of bounding boxes instead of exact pixel-level masks. To tackle this challenge, we introduce a hybrid approach, where we use traditional computer vision techniques, such as Otsu and adaptive thresholding, and marked watershed, to create pixel-level pseudo-masks that are used to train a U-Net model. 
We show that using a small, although fully annotated, dataset is insufficient to train the model. Next, we try training the model on the pseudo-masks created by different computer vision pipelines, on the large weakly annotated dataset. To use the combined strength of different pseudo-masks, we then try making a second generation of them by trying various fusion strategies. Finally, we tried a transfer learning approach, where a model pretrained on a large dataset with pseudo-masks is fine-tuned on a small, fully annotated dataset, and this model achieved the best results. We evaluate each model using quantitative metrics such as the Dice coefficient and the intersection over the union, and qualitative metrics by visualizing its predictions.

% english annotation content here


\newpage{}\thispagestyle{empty}



\thispagestyle{empty}
\section*{Anotácia}

\begin{minipage}[t]{1\columnwidth}%
Slovenská technická univerzita v Bratislave

Fakulta informatiky a informačných technológií

Študijný program: Informatika\\

Autor: \myName

Bakalárska práca: \myTitle

Vedúci bakalárskeho projektu: \mySupervisor

\myDateSK%
\end{minipage}

\bigskip{}

V tejto práci sa zaoberáme technikami slabej segmentácie v digitálnej patológii so zameraním na segmentáciu jadier lymfocytov u pacientov s rakovinou prsníka. Hlavná výzva vyplýva zo slabých anotácií jadier vo forme ohraničujúcich rámčekov namiesto presných masiek na úrovni pixelov. Na riešenie tejto výzvy zavádzame hybridný prístup, kde používame tradičné techniky počítačového videnia, ako sú Otsu a adaptívne prahovanie, a značkami-riadený algoritmus watershed, na vytvorenie pseudo-masiek, ktoré sa používajú na trénovanie modelu U-Net. Ukazujeme, že použitie malého, plne anotovaného datasetu je na trénovanie modelu nedostatočné. Ďalej vyskúšame trénovať model na pseudo-maskách vytvorených rôznymi metódami počítačového videnia na veľkom, slabo anotovanom datasete. Aby sme využili kombinovanú silu rôznych pseudo-masiek, vytvoríme ich druhú generáciu vyskúšaním rôznych stratégií zlúčenia. Nakoniec použijeme prístup učenia s prenosom, kde sa model predtrénovaný na veľkom datasete s pseudo-maskami dotrénuje na malom, plne anotovanom datasete, pričom tento model dosiahol najlepšie výsledky. Každý model hodnotíme pomocou kvantitatívnych metrík, ako sú Dice koeficient a IoU, a kvalitatívnych metrík vizualizáciou jeho predpovedí.

% slovak annotation content here


\newpage{}\thispagestyle{empty}\medskip{}


\newpage{}



\newpage

