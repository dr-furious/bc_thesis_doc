
\thispagestyle{empty}

\section*{Annotation}

\begin{minipage}[t]{1\columnwidth}%
Slovak University of Technology, Bratislava 

Faculty of Informatics and Information Technologies

Degree Course: \myStudyProgram\\

Author: \myName

Bachelor's Thesis: \myTitle

Supervisor: \mySupervisor

\myDate%
\end{minipage}

\bigskip{}

%Breast cancer is one of the leading causes of death among women, and accurate histopathological analysis plays a significant role in diagnosis and treatment planning. Tumor-infiltrating lymphocytes (TILs) have emerged as a promising biomarker. However, manual analysis and counting of TILs in histopathological slides remain time-consuming, error-prone, and require skilled professionals. Deep learning models have shown promise in automating this process, but they depend on large amounts of high-quality annotated data. 

In this thesis, we look at state-of-the-art weak segmentation techniques in digital pathology, focusing on segmenting lymphocyte cell nuclei in breast cancer patients. The main challenge stems from the weak annotations of nuclei in the form of bounding boxes instead of exact pixel-level masks. To tackle this challenge, we introduce a hybrid approach, where we use traditional computer vision techniques, such as Otsu and adaptive thresholding, and marked watershed, to create pixel-level pseudo-masks that are used to train a U-Net model. 

We show that using a small, although fully annotated, dataset is insufficient to train the model. Next, we try training the model on the pseudo-masks created by different computer vision pipelines. To use the combined strength of different pseudo-masks, we then try creating a second generation of them by trying various fusion strategies. Finally, we try a transfer learning approach, where a model pretrained on a large dataset with pseudo-masks is fine-tuned on a small, fully annotated dataset. We evaluate each model on quantitative metrics such as Dice coefficient and Intersection over Union, and qualitative metrics by visualizing its predictions.

% english annotation content here


\newpage{}\thispagestyle{empty}



\thispagestyle{empty}
\section*{Anotácia}

\begin{minipage}[t]{1\columnwidth}%
Slovenská technická univerzita v Bratislave

Fakulta informatiky a informačných technológií

Študijný program: Informatika\\

Autor: \myName

Bakalárska práca: \myTitle

Vedúci bakalárskeho projektu: \mySupervisor

\myDateSK%
\end{minipage}

\bigskip{}

V tejto práci analyzujeme existujúce state-of-the-art techniky a metódy použité pri segmentačných úlohách so slabo anotovanými dátami. Špeciálne sa zameriavame na digitálnu patológiu a tumor-infiltrujúce lymfocyty v nádoroch prsníka. Bunky majú slabé anotácie vo forme tzv. bounding boxov. Navrhujeme metódu vytvárania segmentačných masiek na úrovni pixelov, za použitia tradičných metód počítačového videnia, ako napríklad GrabCut algoritmus, v kombinácii s modelmi hlbokého učenia.

% slovak annotation content here


\newpage{}\thispagestyle{empty}\medskip{}


\newpage{}



\newpage

