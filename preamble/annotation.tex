
\thispagestyle{empty}

\section*{Annotation}

\begin{minipage}[t]{1\columnwidth}%
Slovak University of Technology Bratislava 

Faculty of Informatics and Information Technologies

Degree Course: \myStudyProgram\\

Author: \myName

Bachelor's Thesis: \myTitle

Supervisor: \mySupervisor

\myDate%
\end{minipage}

\bigskip{}

In this work, we analyze the existing state-of-the-art techniques and methods used in weak segmentation tasks, especially focusing on digital pathology and tumor-infiltrating lymphocytes in breast cancer patients. The cells are weakly annotated using bounding boxes, and we propose a method for creating pixel-level masks using traditional techniques of computer vision, such as GrabCut, with a combination of deep learning models.

% english annotation content here


\newpage{}\thispagestyle{empty}

\newpage
\thispagestyle{empty}
\mbox{}
\newpage

\thispagestyle{empty}
\section*{Anotácia}

\begin{minipage}[t]{1\columnwidth}%
Slovenská technická univerzita v Bratislave

Fakulta informatiky a informačných technológií

Študijný program: Informatika\\

Autor: \myName

Bakalárska práca: \myTitle

Vedúci diplomového projektu: \mySupervisor

\myDateSK%
\end{minipage}

\bigskip{}

V tejto práci analyzujeme existujúce state-of-the-art techniky a metódy použité pri segmentačných úlohách so slabo anotovanými dátami. Špeciálne sa zameriavame na digitálnu patológiu a tumor-infiltrujúce lymfocyty v nádoroch prsníka. Bunky majú slabé anotácie vo forme tzv. bounding boxov. Navrhujeme metódu vytvárania segmentačných masiek na úrovni pixelov, za použitia tradičných metód počítačového videnia, ako napríklad GrabCut algoritmus, v kombinácii s modelmi hlbokého učenia.

% slovak annotation content here


\newpage{}\thispagestyle{empty}\medskip{}


\newpage{}

\newpage
\thispagestyle{empty}
\mbox{}
\newpage



\newpage

